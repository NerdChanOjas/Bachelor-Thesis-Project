Due to the advancement of technology in the society especially in the healthcare sector, it has become very important for the patients as well as providers to have easy to easy access to medical information and to be able to manage it effectively. People find it difficult to comprehend their complicated medical records including prescriptions, laboratory results and treatment regimens that results to confusion that may lead to wrong health decisions. In contrast, doctors and administrative employees engaging in computing these large patient data obtained from the EHRs or any affiliated computer systems have to spend considerable time in data search and processing with high chances of making errors. These challenges posit the importance of an intelligent solution that will help to disengage the gap between the patients and the healthcare providers via a more simplified and accurate means to seek for medical information.

As for now with the help of LLMs, such as GPT-4 and Retrieval-Augmented Generation (RAG), the AI systems can produce responses relevant to the context with the help of an external knowledge base. The described models can become a significant improvement for the communication process in the field of healthcare since they can offer immediately and always correct solutions to patients' questions as well as help healthcare workers in finding the needed information. However, existing Systems still have some limitations like hallucinations meaning, the generation of false information, and some problems, which concern the security and the protection of the users' rights and the data in compliance with the rules of organizations like HIPAA, GDPR\. To this end, this thesis suggests the design and implementation of a RAG-based LLM system suitable for healthcare context with a goal to improve the patient's and the provider's satisfaction, as well as protect the privacy and confidentiality of the medical data.