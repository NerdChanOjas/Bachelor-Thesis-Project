\section{General Introduction}
Due to the advancement of technology in society, especially in the healthcare sector, it has become very important for the patients as well as providers to have easy access to medical information and to be able to manage it effectively. People find it difficult to comprehend their complicated medical records, including prescriptions, laboratory results, and treatment regimens, which results in confusion that may lead to wrong health decisions. In contrast, doctors and administrative employees engaging in computing this large patient data obtained from the Electronic Health Records (EHR) have to spend considerable time in data search and processing with high chances of making errors.

As for now, with the help of LLMs, such as LLaMA or Mistral and Retrieval-Augmented Generation (RAG), the AI systems can produce responses relevant to the context with the help of an external knowledge base. The described models can make a significant improvement for the communication process in the field of healthcare since they can offer immediate and correct solutions to patients' questions as well as help healthcare workers in finding the needed information. However, existing systems still have some limitations, like hallucinations, meaning the generation of false information, and some other problems, which concern the security and protection of the users' rights and the data in compliance with the rules of DISHA\cite{Ganapathy} and DPDPA\cite{SundaraNarendran+2023+129+141}. To the end, this thesis suggests a design and implementation of a RAG-based LLM system suitable for healthcare context with a goal to improve the patient's and the provider's satisfaction, as well as protect the privacy and confidentiality of the medical data.

\newpage

\section{Motivation}
\indent The answer to the origin of this motivation comes from the fact that healthcare data are becoming complex and there is a need for a reliable and responsible AI-driven system to process these data. Some recent studies on RAG systems show that the external knowledge integration capability of RAG systems can enhance the performance of LLM by minimising hallucinations and maximising the response accuracy. But they are yet to be installed in real-life applications in a healthcare system or where patients' lives are at risk, and if the information given is inaccurate, it can lead to harm.

The patient's records are often incomprehensible to the particular patient, which may, in turn, entail wrong decisions about his/her treatment and wellbeing. Health care providers, on the other hand, require quicker and more efficient means of retrieving and analysing patient data, especially in emergency situations. By combining LLMs with a more robust RAG framework, these issues are not as much of a hurdle because they provide users with context-specific information. Furthermore, when human-in-the-loop strategies were incorporated into the system, it was possible to attain procedural accuracy and clinical validity, thus making the technology appropriate for sensitive healthcare applications.

\newpage

\section{Problem Statement}
Modern healthcare organisations deal with massive volumes of data of various types, ranging from patient records to lab results. Patients struggle to understand what doctors tell them about themselves or what test results mean. They too have trouble making sense of the information presented to them. On the other hand, healthcare providers struggle to manage, retrieve, analyse the data quickly. There is a major missing link in terms of ease of use, time and context sensitivity of the current systems available for not only the patient but also the clinician. In the current system, they do lack both contextual awareness and accuracy, which are present in the current chatbots and digital healthcare solutions.

Even though models including LLaMA are accurate in using language to provide health information, they have problems with hallucinations and facts. These drawbacks have been considered to be tackled by the adoption of RAG systems since they provide additional knowledge to the generation process. However, some issues arise, like keeping the privacy of data, protecting patient data, and designing a system that can conform to current medical databases. In the scope of this thesis, the following are the challenges that this project is aimed to solve: The RAG-LLM system proposed in this project intends to come up with a reliable, scalable prototype that offers relevant, context-aware information to patients and care providers while satisfying legal requirements such as those presented by DISHA\cite{Ganapathy} and DPDPA\cite{SundaraNarendran+2023+129+141}.