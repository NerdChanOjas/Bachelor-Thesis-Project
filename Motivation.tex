The answer to the origin of this motivation comes from the fact that the healthcare data are becoming complex and there is need for reliable AI-driven system to process these data. Some recent studies on RAG systems show that, the external knowledge integration capability of RAG systems can enhance the performance of LLM by minimizing hallucinations and maximizing the response accuracy. But, they are yet to be installed in real life applications in a health care system or where patients' lives are at risk and it the information given is in accurate can lead to harm.

The patient's records are often incomprehensible to the particular patient which may, in turn, entail wrong decisions about his/her treatment and wellbeing. Health care providers on the other hand require quicker and efficient means of retrieving and analyzing patient data most especially in emergency situations. By combining LLMs like GPT-4 and LLaMA models with a more robust RAG framework, these issues are not as much of a hurdle because they avail users with context-specific information. Furthermore, when human-in-the-loop strategies were incorporated into the system, it was possible to attain procedural accuracy and clinical validity thus making the technology appropriate for sensitive health-care applications.