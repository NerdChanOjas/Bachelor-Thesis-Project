The use of Retrieval-Augmented Generation (RAG) together with Large Language Models (LLMs) within the healthcare field is one of the current trends, which has been the centre of attention in the recent years mainly because of the issues connected with the quality as well as context comprehension of the medical data. Haez et al. (2024) have suggested improved RAG strategy to increase the trust level in the medical chatbot by adding an initial interaction cycle in the RAG pipeline. This involves the LLM creating a mock document to use in requesting from a certified information source hence minimizing hallucinations in responses. Their work also shows that despite the current challenges, RAG-LLMs can improve the user trust specifically in maternal health domains by using the certified knowledge sources for the responses\cite{10.1007/978-3-031-66538-7_22}.

In the same vein, Al Ghadban et al., (2023) discuss the feasibility of using RAG models in healthcare education learning with frontline health workers in LMICs. One tool developed by them is known as “SMARThealth GPT”\cite{wu2024medicalgraphragsafe} that employ RAG to generate targeted, context-sensitive information to foster comprehension of the existing gaps in the delivery of community health services. It supports the RAG's capacity in individually catering LLMs for expanding educational ends, in boosting the health worker's ability on accurate guideline-based care. Furthermore, another study is about using generative AI with RAG to derive the critical clinical data from the EHRs. This way patient data summarization is performed and examples are shown on how a RAG system can reduce the burden of data management on clinicians while at the same time providing them with context relevant information\cite{paper2}.

Comparing the medical application of RAG also have a significant part to investigate its performance. Studies that compared RAG in requiring the healthcare domain present the advantage and drawbacks of using LLMs for searching for medical information. Therefore, this work lays a foundation for the development of future implementations in healthcare by raising the element of the need to include retrieval mechanisms that will enable the delivery of contextually relevant and precise information\cite{xiong2024benchmarkingretrievalaugmentedgenerationmedicine}. Research on the effectiveness of EMR search engines continues to indicate that learning to rank techniques have the potential of further boosting RAG systems' performance in dealing with the large volumes of medical information. This study shows that augmenting learning-to-rank approaches can enhance the process of document search and enhance patient treatment by offering better outcomes\cite{YE202493}.

The other significant area appropriate for RAG-LLM development is dealing with the issue of semantic uncertainty and making the answers more accurate. Query-based innovations in RAG systems are described in a study but the approaches employed in the study are applied in reducing ambiguity and enhancing the retrieved documents relevance. This way, the given approach enhances the validity of LLM-produced responses in medical situations, which should bring the enhanced trust of the users in automatized healthcare systems\cite{yang2024geometryqueriesquerybasedinnovations}. To supplement the reliability of medical chatbots, SelfRewardRAG\cite{10620139} provides LLMs with a self-evaluation function so as to enable them critically evaluate their generated responses in terms of accuracy and relevancy. This helps in minimizing the frequency of hallucinations and improves the quality of generated responses demonstrating that self-evaluation can effectively cause enhanced LLM performance in the medical context. There is also some emerging safety issues which have also been discussed in the recent papers as applying AI to generate medical advice. The use of graph-based RAG systems in one study therefore trains the system with rules and regulation that will make the LLM utterances conform to certified medical standard. The use of graph retrieval techniques improves on the safety and accuracy of the interactions and especially on patients' sensitive data. This emphasis on using verified information sources show the key idea of developing safe and rather reliable artificial intelligence applications in healthcare. paper 568