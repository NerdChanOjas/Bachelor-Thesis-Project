It have been noticed that in the present day's health care delivery system the communication between patients and healthcare providers is of paramount important. One of the main issues, which arise dealing with medical information, including electronic health records (EHRs), laboratory results, and prescriptions, is that all the information can be overwhelming. This results to producing confusion and wrong decisions regarding their health. On the other hand, the health care providers are experiencing challenges on how to address the big issue of managing and searching large amount of data from multiple hospital information systems using repetitive and manual methods which are inconveniencing. Such existing and currently popular tools, and chatbots, for instance, provide simple solutions but cannot give a context-aware answer in real-time for both the patient and the health care personnel.

Recently, Large Language Models (LLMs) and Retrieval-Augmented Generation (RAG) systems present the solution path to these problems. Although the investigated LLMs are capable of providing natural language outputs, they are inclined to certain problems such as hallucinations. RAG systems therefore propose to improve the reliability of these models by incorporating external sources of knowledge for production of accurate and reliable results. The study reveals that RAG systems lead to noteworthy reduction in hallucination and increase in quality of the response as the system incorporates only certified medical knowledge to derive response and such complex health care sectors like maternal care are most vulnerable to benefit from the implementation of RAG systems.

Nevertheless, there are difficulties in the development of RAG-LLM systems that should support the calculation of the result in healthcare, making the system as reliable as possible, and at the same time, consuming minimal resources. Some of the key stakeholders' concerns include privacy of the data, accuracy of medical information in the system, and integration of the new system management with the existing hospital management systems. However, there is a need for human-in-the-loop approaches to handle the vagueness as well as to make sure that the last decisions are made by doctors. This paper aims at developing a RAG based LLM system in healthcare context with special reference to both patient and provider interfaces. The solution seeks to reduce the patient's involvement as much as possible with their records while at the same time making the retrieval of information smooth, fast and efficient for the healthcare providers; it makes communication in the health sector safe and reliable.