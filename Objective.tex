The primary goal of this thesis is to employ Retrieval-Augmented Generation (RAG) and Large Language Models (LLM) to build an AI-assisted healthcare communication system to improve the experience of the patient or healthcare provider relationship. The points for the objective are as follows:

\begin{itemize}
    \item Improve Patient Access to Medical Information: Develop a chatbot that allows patients to interact with their medical data in simple, natural language, providing clear and accurate responses to queries about lab reports, prescriptions, and treatment plans.
    \item Streamline Data Retrieval for Healthcare Providers: Create a system that enables healthcare workers to retrieve patient data quickly and efficiently through natural language queries, reducing the time spent on manual searches and improving decision-making.
    \item Mitigate Hallucinations and Enhance Reliability: Integrate RAG systems to augment the LLM generation process with certified external knowledge, minimizing the risk of hallucinations and ensuring trustworthy, accurate responses.
    \item Ensure Data Privacy and Compliance: Implement robust encryption, access control mechanisms, and audit logs to secure patient data and ensure compliance with healthcare regulations like HIPAA\cite{hipaa} and GDPR\@.
    \item Incorporate Human-in-the-Loop Mechanisms: Introduce a human-in-the-loop mechanism for reviewing and validating ambiguous or critical queries to ensure the highest level of accuracy in the system's responses.
\end{itemize}