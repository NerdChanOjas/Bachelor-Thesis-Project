\section{Prior Work}
This section discusses the prior work done in the support of LLMs with RAGs for hospital based environment and how the concept evolved from this area and how can this help in making a solution that can help the heathcare interaction system simplify there process.

New trends in medical informatics have incorporated the use of the graph and the RAG frameworks as approaches to enhance information retrieval and question answering in the medical field. A paper proposed the MedGraphRAG framework\cite{wu2024medicalgraphragsafe}, which utilizes a hierarchical graph model to enhance the retrieval process by linking medical entities across three tiers: by the data furnished by the patient, basic medical knowledge, and medical encyclopedias or dictionaries respectively.  The U-retrieve method which they used does a good job at handling a balance between global context and indexing efficiency, which in the end gives better results.

The latter domain was further explored in the next work, which developed the MIRAGE benchmark\cite{xiong2024benchmarkingretrievalaugmentedgenerationmedicine} for field-testing the performance of the RAG system in the context of medical question-answering. This showcases the potential of RAG when it comes to creating high-class retrieval pipelines, precisely for medical applications. So when working in a high-class retrievals pipelines for medical application, this showed the potential of RAG\@.

Another notable contribution, who developed a learning-to-rank approach\cite{YE202493} aimed at improving RAG-based search engines for Electronic Medical Records (EMR). Their system adapts to user search semantics by learning from user feedback, adjusting the ranking of retrieved documents based on relevance, which significantly improves retrieval accuracy in EMR systems.

The SelfRewardRAG method\cite{10620139} introduced a self-evaluation layer into the retrieval process. This technique enables large language models (LLMs) to reflect on and refine their responses based on past outputs. By synthesizing information from multiple sources and applying chain-of-thought reasoning, SelfRewardRAG enhances the safety and accuracy of medical reasoning.

To address misalignment issues between user queries and retrieved content, Yang and other introduced Query-Based RAG (QB-RAG)\cite{yang2024geometryqueriesquerybasedinnovations}. This system generates a comprehensive set of pre-defined queries, helping to bridge the gap between user questions and the relevant content, thereby improving retrieval accuracy in handling complex medical queries.

In the realm of Electronic Health Records (EHRs), a paper demonstrated the application of generative AI combined with RAG for summarizing and extracting key clinical information\cite{paper2}. Their system achieved an impressive high accuracy in identifying risk factors, showcasing RAG's potential for managing large volumes of structured and unstructured clinical data.

Further refining the interaction between retrieval and generation, the Iterative Retrieval-Generation (I-RetGen)\cite{10.1007/978-3-031-66538-7_22} method integrates these processes iteratively, with each generation step guiding more precise retrieval results. This approach is especially useful for handling complex medical queries, where successive refinement leads to more accurate responses.

Lastly, FLARE (Active Retrieval)\cite{10.1007/978-3-031-66538-7_22} introduces a dynamic aspect to the retrieval process by determining in real time when and what to retrieve during generation. This adaptability allows RAG systems to adjust to evolving user queries more effectively, making FLARE particularly valuable in dynamic medical environments.

\newpage

\section{Objectives}
The primary goal of this thesis is to employ Retrieval-Augmented Generation (RAG) and Large Language Models (LLM) to build an AI-assisted healthcare communication system to improve the experience of the patient or healthcare provider relationship. The points for the objective are as follows:

\begin{itemize}
    \item Improve Patient Access to Medical Information: Develop a chatbot that allows patients to interact with their medical data in simple, natural language, providing clear and accurate responses to queries about lab reports, prescriptions, and treatment plans.
    \item Streamline Data Retrieval for Healthcare Providers: Create a system that enables healthcare workers to retrieve patient data quickly and efficiently through natural language queries, reducing the time spent on manual searches and improving decision-making.
    \item Mitigate Hallucinations and Enhance Reliability: Integrate RAG systems to augment the LLM generation process with certified external knowledge, minimizing the risk of hallucinations and ensuring trustworthy, accurate responses.
    \item Ensure Data Privacy and Compliance: Implement robust encryption, access control mechanisms, and audit logs to secure patient data and ensure compliance with healthcare regulations like DISHA\cite{Ganapathy} and DPDPA\@\cite{SundaraNarendran+2023+129+141}.
    \item Incorporate Human-in-the-Loop Mechanisms: Introduce a human-in-the-loop mechanism for reviewing and validating ambiguous or critical queries to ensure the highest level of accuracy in the system's responses.
\end{itemize}